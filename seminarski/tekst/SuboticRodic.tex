\documentclass[12pt]{article}
\usepackage[utf8]{inputenc}
\usepackage[T1]{fontenc}

\begin{document}

\title{Algoritmi za detekciju plagijarizma 1 (opis i implementacija) \\ \normalsize seminarski rad u okviru kursa Metodologija stručnog i naučnog rada \\ \normalsize Matematički fakultet, Beograd}
\author{Nemanja Subotić i Igor Rodić}
\date{26. Mart, 2016.}
\maketitle

\begin{center}
\textbf{Predgovor}
\end{center}
\par Sa pojavom modernih tehnologija, pre svega interneta, svet je postao u velikoj meri izložen raznim oblicima zloupotreba. Jednu od tih zloupotreba predstavljaju plagijarizmi. Zbog ogromne količine podataka koja je dostupna velikoj većini ljudske civilizacije, lakše je nego ikada naći potrebne informacije o bilo čemu što nas zanima, ali to sa sobom nosi i rizike. Jer skoro sa istom lakoćom bilo ko može te informacije da iskopira (preformuliše, ukrade ideju) i prezentuje kao svoje delo. Prirodno je potražiti odgovor na ovo pitanje u kompjuterskim algoritmima, jer bi taj zadatak za čoveka bio previše obiman. \newline
\par Jedan od oblika detekcije plagijarizama je i detekcija plagijarizma programskog koda, koji je za nas informatičare posebno interesantan, i njime ćemo se baviti u ovom radu.

\renewcommand*\contentsname{Sadržaj}
\tableofcontents
\section{Uvod}
Ovo je uvod.
\section*{Literatura}

\end{document}

\end{document}